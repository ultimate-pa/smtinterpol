\documentclass[a4paper]{easychair}
\usepackage[english]{babel}
\usepackage{xspace}
\usepackage{hyperref}
\usepackage{bashful}

\newcommand\SI{SMTInterpol\xspace}
\newcommand\SIrem{SMTInterpol-remus\xspace}
\newcommand{\version}{\splice{git describe} and\splice{git describe MUSes-smtcomp2021}}
\newcommand{\TODO}[1]{\textcolor{red}{#1}}

\title{\SI and \SIrem\\{\large Versions\version}}

\author{Leonard Fichtner \and Jochen Hoenicke \and Moritz Mohr \and Tanja Schindler}
\institute{
  University of Freiburg\\
  \email{\{hoenicke,schindle\}@informatik.uni-freiburg.de}\\[1ex]
  \today
}
\titlerunning{\SI \version}
\authorrunning{Fichtner, Hoenicke, Mohr, and Schindler}

\begin{document}
\maketitle
\section*{Description}
\SI is an SMT solver written in Java and available under LGPL v3.
It supports the combination of the theories of uninterpreted functions, linear arithmetic over integers and reals, arrays, and datatypes.
Furthermore it can produce models, proofs, unsatisfiable cores, and interpolants.
The solver reads input in SMT-LIB format.
It includes parsers for DIMACS, AIGER, and SMT-LIB version 2.6.

The solver is based on the well-known DPLL(T)/CDCL framework~\cite{DBLP:conf/cav/GanzingerHNOT04}.
It uses variants of standard algorithms for CNF conversion~\cite{DBLP:journals/jsc/PlaistedG86} and congruence closure~\cite{DBLP:conf/rta/NieuwenhuisO05}.
The solver for linear arithmetic is based on Simplex~\cite{DBLP:conf/cav/DutertreM06}, the sum-of-infeasibility algorithm~\cite{DBLP:conf/fmcad/KingBD13}, and branch-and-cut for integer arithmetic~\cite{DBLP:conf/cav/ChristH15,DBLP:conf/cav/DilligDA09}.
The array decision procedure is based on weak equivalences~\cite{DBLP:conf/frocos/ChristH15} and includes an extension for constant arrays~\cite{DBLP:conf/vmcai/HoenickeS19}.
The solver for quantified formulas performs an incremental search for conflicting and unit-propagating instances of quantified formulas~\cite{DBLP:conf/vmcai/HoenickeS21} which is complemented with a version of enumerative instantiation~\cite{DBLP:conf/tacas/ReynoldsBF18} to ensure completeness for the finite almost uninterpreted fragment~\cite{DBLP:conf/cav/GeM09}.
Theory combination is performed based on partial models produced by the theory solvers~\cite{DBLP:journals/entcs/MouraB08}.

This year, a solver for datatypes has been added.
It is based on the rules presented in~\cite{DBLP:journals/jsat/BarrettST07}.
SMTInterpol has also been extended with unsatisfiable core enumeration based on the ReMUS algorithm~\cite{DBLP:conf/atva/BendikCB18}.
The extension allows for choosing a best unsatisfiable core according to several heuristics.

The main focus of \SI is the incremental track.
This track simulates the typical application of \SI where a user asks multiple queries.
As an extension \SI supports quantifier-free interpolation for all supported theories except for datatypes~\cite{DBLP:journals/jar/ChristH16,DBLP:conf/cade/HoenickeS18,DBLP:conf/vmcai/HoenickeS19}.
This makes it useful as a backend for software verification tools. 
In particular, \textsc{Ultimate Automizer}\footnote{\url{https://ultimate.informatik.uni-freiburg.de/}} and \textsc{CPAchecker}\footnote{\url{https://cpachecker.sosy-lab.org/}}, the winners of the SV-COMP 2016--2021, use \SI.

\section*{Competition Versions}
The version of \SI submitted to the SMT-COMP 2021 contains the new solver for datatypes.
A peculiarity in this year's versions is that the proof check mode is enabled, which means that proofs of unsatisfiability are tracked and checked by the internal proof checker (an exception are lemmas of the theory of datatypes).

\SIrem is a variant of \SI that uses an enumeration algorithm in order to find smaller unsatisfiable cores if possible.

\section*{Webpage and Sources}
Further information about \SI can be found at
\begin{center}
  \url{https://ultimate.informatik.uni-freiburg.de/smtinterpol/}
\end{center}
The sources are available via GitHub
\begin{center}
  \url{https://github.com/ultimate-pa/smtinterpol}
\end{center}

\section*{Authors}
The code was written by J{\"u}rgen Christ, Daniel Dietsch, Leonard Fichtner, Matthias Heizmann, Jochen Hoenicke, Moritz Mohr, Alexander Nutz, Markus Pomrehn, Pascal Raiola, and Tanja Schindler.

\section*{Logics, Tracks and Magic Number}

\SI participates in the single query track, the incremental track, the unsat core track, and the model validation track.
It supports all combinations of uninterpreted functions, linear arithmetic, arrays, and datatypes, and participates in the following logic divisions:

ALIA, ANIA, AUFLIA, AUFLIRA, AUFNIRA, LIA, LRA,
QF\_ALIA, , QF\_ANIA, QF\_AUFLIA, QF\_AX, QF\_DT, QF\_IDL, QF\_LIA, QF\_LIRA, QF\_LRA, QF\_NIA, QF\_RDL, QF\_UF, QF\_UFDT, QF\_UFDTLIRA, QF\_UFIDL, QF\_UFLIA, QF\_UFLRA, QF\_UFNRA, QF\_UFNIA,
UF, UFIDL, UFLIA, UFLRA, UFNIA, UFNRA.

\SIrem participates in the quantifier-free divisions of the unsat core track:
QF\_ALIA, QF\_AUFLIA, QF\_AUFNIA, QF\_AX, QF\_IDL, QF\_LIA, QF\_LIRA, QF\_LRA, QF\_RDL, QF\_UF, QF\_UFIDL, QF\_UFLIA, QF\_UFLRA
\TODO{quantifiers, datatypes?}

Magic Numbers: 384076683+1

\bibliography{sysdec}
\bibliographystyle{alpha}
\end{document}
