%%%%%%%%%%%%%%%%%%%%%%%%%%%%%%%%%%%%%%%%%%%%%%%%%%%%%%%%%%%%%%%%%%%%%%%%
%
%   Folien: SMT Workshop
%           SMTInterpol
%   Juni/Juli 2012
%
%   Vortragsdauer: ??+5 Minuten
%
%%%%%%%%%%%%%%%%%%%%%%%%%%%%%%%%%%%%%%%%%%%%%%%%%%%%%%%%%%%%%%%%%%%%%%%%
%
%  - SMTInterpol (logics, unsat core, interpolants)
%

\documentclass[table]{beamer}
\usepackage{xcolor}
\usepackage{tikz}
%\usepackage{jharrows}
\usetikzlibrary{fit,shapes.callouts,shapes.geometric,shapes.symbols}
\usetikzlibrary{positioning,calc}
\usetheme{Madrid}
\usecolortheme{freiburg}
\let\checkmark\relax
\usepackage{dingbat}

\newcommand\store[3]{\ensuremath{#1\langle #2\lhd#3\rangle}}
\newcommand\select[2]{\ensuremath{#1[#2]}}
\newcommand\diff[2]{\ensuremath{\mathop{\mathrm{diff}}(#1,#2)}}

% The face style, can be changed
\tikzset{face/.style={shape=circle,minimum size=20ex,shading=radial,outer sep=0pt,
        inner color=white!50!yellow,outer color= yellow!70!orange}}
%% Some commands to make the code easier
\newcommand{\emoticon}[1][]{%
  \node[face,#1] (emoticon) {};
  %% The eyes are fixed.
  \draw[fill=white] (-1ex,0ex) ..controls (-0.5ex,0.2ex)and(0.5ex,0.2ex)..
        (1ex,0.0ex) ..controls ( 1.5ex,1.5ex)and( 0.2ex,1.7ex)..
        (0ex,0.4ex) ..controls (-0.2ex,1.7ex)and(-1.5ex,1.5ex)..
        (-1ex,0ex)--cycle;}
\newcommand{\pupils}{
  %% standard pupils
  \fill[shift={(0.5ex,0.5ex)},rotate=80] 
       (0,0) ellipse (0.3ex and 0.15ex);
  \fill[shift={(-0.5ex,0.5ex)},rotate=100] 
       (0,0) ellipse (0.3ex and 0.15ex);}

\geometry{papersize={16.80cm,10.24cm}}
\setbeamertemplate{navigation symbols}{}

%%% Titel, Autor und Datum des Vortrags:
\pgfdeclareimage[height=1cm]{unifr}{unifr-neu}
\title{SMTInterpol}
\author[J{\"u}rgen Christ]{J{\"u}rgen Christ \and Jochen Hoenicke \and Alexander Nutz}
\date[SMT Workshop]{SMT Workshop}

%%% Logos
%\pgfdeclareimage[height=.84cm]{university-logo}{Uni-Logo-kompakt}
%\logo{\pgfuseimage{university-logo}}

%\pgfdeclareimage[height=1cm]{csd-logo}{csd-logo-text}
%\logo{\hskip.2cm\pgfuseimage{csd-logo}}

%% Institut
\institute[Uni Freiburg]{University of Freiburg\hspace{1cm}\pgfuseimage{unifr}}

\colorlet{colorTest}{ALUblue}
\colorlet{colorLamps1}{orange!70!black}
\colorlet{colorLamps2}{orange!70!black}
\colorlet{callOut}{red!30!white}
\colorlet{calledOut}{ALUgray}

\useoutertheme{unifr}


\begin{document}
%\renewcommand{\inserttotalframenumber}{25}


%%% Titelfolie malen:
%% \begin{frame}
%% \titlepage
%% \end{frame}
%{
%  \setbeamercolor{background canvas}{bg=}
%  \includepdf[pages=1]{1103_FASE2011-4.pdf}
%}
%\refstepcounter{framenumber}

\section{Motivation}

\begin{frame}[fragile]
  \pgfdeclarelayer{back}
  \pgfsetlayers{back,main}
  \begin{tikzpicture}
    \uncover<2->{
    \begin{scope}[xshift=-5cm,yshift=4.5cm]
      \uncover<2-3>{
        \node (qfla) at (0,.4) {Quantifier Free};
      }
      \node (la) at (0,0) {Linear Arithmetic};
      \node (la1) at (.5,-.8) {$y \leq i+1$};
      \node (la2) at (-.8,-1.3) {$i \leq y$ };
      \uncover<3-> {
        \node (lamixed) at (0,-1.8) {$y - to\_int(y) < .3$};
      }
      \node<3>[draw,shape=rectangle callout,
        callout absolute pointer={(lamixed.south)}]
        at (-1,-3.5) {mixed int/real};
      \node<3>[starburst, fill=yellow, draw=red, line width=2pt] at
           (2, -3.5) {\bf NEW};
    \end{scope}
    \begin{scope}[xshift=5cm,yshift=4.5cm]
      \uncover<2-3>{
        \node (qfuf) at (0,.4) {Quantifier Free};
      }
      \node (uf) at (0,0) {Uninterpreted Functions};
      \node (uf1) at (-.6,-.8) {$f(b) = v$};
      \node (uf2) at (.7,-1.3) {$f(a) \neq v$ };
    \end{scope}
    \begin{scope}[xshift=0cm,yshift=5cm]
      \uncover<2->{
        \node (qfar) at (0,.4) {Quantifier Free};
      }
      \node (ar) at (0,0) {Arrays};
      \node (ar1) at (-.3,-.8) {$b = \store aiv$};
      \node (ar2) at (.3,-1.3) {$\select av = v$};
    \end{scope}
    }
    \uncover<4->{
      \begin{scope}[xshift=0cm,yshift=4cm]
        \node (lauf1) at (0, -1.5) {$\select b i \geq i$};
        \node (lauf2) at (-.7, -2) {$f(i+y) = 2v$};
    \end{scope}}
    \uncover<7->{
      \begin{scope}[xshift=0cm,yshift=4cm]
        \node (lauf3) at (.3, -2.5) {$f(b) \leq i$};
    \end{scope}}
    \uncover<1-5>{
      \node[align=left] at (0,-1.5) {
        decides {\bf S}atisfiability {\bf M}odulo {\bf T}heory\\
        computes Craig {\bf Interpol}ants};
    }
    \uncover<2-3>{
      \node[shape=cloud callout,cloud puffs=20,draw,inner sep=0cm,aspect=2,
        callout relative pointer={(.5,-.7)},fit=(la)(la1)(la2)] {};
      \node[shape=cloud callout,cloud puffs=20,draw,inner sep=0cm,aspect=2,
        callout relative pointer={(-.5,-.7)},fit=(uf)(uf1)(uf2)] {};
      \node[shape=cloud callout,cloud puffs=20,draw,inner sep=0cm,aspect=2,
        callout relative pointer={(0,-.7)},fit=(ar)(ar1)(ar2)] {};
    }
    \node<2-3>[starburst, fill=yellow, draw=red, line width=2pt] at
       (0, 2.4) {\bf 2014};
    \uncover<4->{
      \coordinate (a1) at (-6,3.6);
      \coordinate (a2) at (6.5,3.6);
      \node[shape=cloud callout,cloud puffs=40,draw,inner sep=-1cm,aspect=3,
        callout relative pointer={(0,-.4)},
        fit=(a1)(a2)
      ] {};
    }
    \uncover<9>{
      \node[shape=ellipse,inner sep=-.12cm,fit=(ar1),draw=ALUred,thick]{};
      \node[shape=ellipse,inner sep=-.12cm,fit=(ar2),draw=ALUred,thick]{};
      \node[shape=ellipse,inner sep=-.12cm,fit=(uf1),draw=ALUred,thick]{};
      \node[shape=ellipse,inner sep=-.12cm,fit=(uf2),draw=ALUred,thick]{};
      \node[shape=ellipse,inner sep=-.12cm,fit=(lauf1),draw=ALUred,thick]{};
      \node[shape=ellipse,inner sep=-.12cm,fit=(lauf3),draw=ALUred,thick]{};
    }
    \begin{pgfonlayer}{back}
      \uncover<10->{
        \node[shape=ellipse,inner sep=-.12cm,fit=(ar1),fill=ALUred!20!white]{};
        \node[shape=ellipse,inner sep=-.12cm,fit=(ar2),fill=ALUred!20!white]{};
        \node[shape=ellipse,inner sep=-.12cm,fit=(uf2),fill=ALUred!20!white]{};
        \node[shape=ellipse,inner sep=-.12cm,fit=(uf1),fill=ALUgreen!20!white]{};
        \node[shape=ellipse,inner sep=-.12cm,fit=(lauf1),fill=ALUgreen!20!white]{};
        \node[shape=ellipse,inner sep=-.12cm,fit=(lauf3),fill=ALUgreen!20!white]{};
      }
    \end{pgfonlayer}
    \node at (1.75,-.1) {\pgfuseimage{unifr}};
    \node (si) at (0,0) {\huge SMTInterpol};
    
    \uncover<5-6>{
      \node[draw,shape=rectangle callout, callout relative pointer={(1.7,-.3)}] 
      at (-4,.6) {sat};
    }
    \uncover<6>{
      \node[draw,shape=rectangle callout, callout relative pointer={(1.4,.2)}]
      (model) at (-4, -.4) {model};
%      \node[starburst, fill=yellow, draw=red, line width=2pt] at
%           ($(model.south)-(0,.7)$) {\bf NEW};
    }

    \uncover<7->{
      \node[draw,shape=rectangle callout, callout relative pointer={(-3.2,-.5)}] 
      at (5.4,1.2) {unsat};
    }
    \uncover<8->{
      \node[shape=ellipse,inner sep=.01cm,thick](proof) at (5.4,.35) {proof};
      \node[draw,shape=rectangle callout, fit=(proof),
        callout relative pointer={(-2.5,-.15)}] {};
    }
    \uncover<9->{
%      \alt<9>{
%      \node[shape=ellipse,inner sep=.03cm,draw=ALUred,thick] at (4.4,-.55) {uns%at core};
%      }%
      {
      \node[shape=ellipse,inner sep=.03cm,thick](uc) at (5.4,-.55) {unsat core};
      }
      \node[draw,shape=rectangle callout, fit=(uc),
        callout relative pointer={(-1.8,.1)}] (ucco) {};
%      \uncover<10>{
%        \node[starburst, fill=yellow, draw=red, line width=2pt] at
%        ($(ucco.south)-(0,.7)$) {\bf NEW};
%      }
    }
    \uncover<10->{
      \node (ips) at (0,-1.5) {interpolants};
      \node (a1) at (-3,-2) {\color{ALUred}$b = \store aiv$};
      \node (a2) at (-3,-2.5) {\color{ALUred}$\select av = v$};
      \node (a3) at (-3,-3) {\color{ALUred}$f(a) \neq v$};
      \node at(-1,-2.5) {\LARGE$\Rightarrow$};
      \node[align=center] (i) at (0.3,-2.5) {$i \neq v$ \\ $\lor$\\$f(b) \neq v$};
      \node at(1.8,-2.5) {\LARGE$\Rightarrow~\lnot$};
      \node (b1) at (3.6,-2) {\color{ALUgreen}$f(b) = v$};
      \node (b2) at (3.6,-2.5) {\color{ALUgreen}$\select bi \geq i$};
      \node (b3) at (3.6,-3) {\color{ALUgreen}$f(b) \leq i$};
      \node[draw,shape=rectangle callout, callout relative pointer={(0,+.8)},
            fit=(ips)(a1)(a2)(a3)(b1)(b2)(b3)(i)] {};
    }
    % Here, we magnify the improvements
%    \uncover<2>{
%      \node[magnifying glass,line width=1ex,draw,fill=yellow!80!red] (cm) at (0,2)
%           {clause minimization};
%           \node[starburst, fill=yellow, draw=red, line width=2pt] at
%           (cm.north) {\bf NEW};
%    }
    %% \uncover<4>{
    %%   \node[magnifying glass,line width=1ex,draw,fill=yellow!80!red] (ps) at (-4,1)
    %%        {pivot strategy};
    %%        \node[starburst, fill=yellow, draw=red, line width=2pt] at
    %%        (ps.north) {\bf NEW};
    %% }
    %% \uncover<12>{
    %%   \node[magnifying glass,line width=1ex,draw,fill=yellow!80!red] (ip) at (-4,-1)
    %%        {QF\_UFLIA interpolation};
    %%        \node[starburst, fill=yellow, draw=red, line width=2pt] at
    %%        (ip.north) {\bf NEW};
    %% }
  \end{tikzpicture}
\end{frame}

\begin{frame}[fragile]
  \frametitle{SMTInterpol is $\ldots$}
  \pgfdeclarelayer{back}
  \pgfsetlayers{back,main}
  \begin{tikzpicture}
    \begin{pgfonlayer}{back}
      \begin{scope}[scale=5]
        \emoticon
        %% pupils up
        \fill<1>[shift={( 0.5ex,1.1ex)},rotate= 80] (0,0) ellipse (0.15ex and
        0.15ex);
        \fill<1>[shift={(-0.5ex,1.1ex)},rotate=100] (0,0) ellipse (0.15ex and
        0.15ex);
        %% pupils right
        \fill<2>[shift={( 0.95ex,0.5ex)},rotate=80] (0,0) ellipse (0.3ex and
        0.15ex);
        \fill<2>[shift={(-0.25ex,0.5ex)},rotate=80] (0,0) ellipse (0.3ex and
        0.15ex);
        %% pupils right down
        \fill<3>[shift={( 0.85ex,0.3ex)},rotate= 80] (0,0) ellipse (0.15ex and
        0.15ex);
        \fill<3>[shift={(-0.25ex,0.35ex)},rotate=100] (0,0) ellipse (0.15ex and
        0.15ex);
        %% pupils down
        \fill<4->[shift={( 0.5ex,0.25ex)},rotate= 80] (0,0) ellipse (0.15ex and
        0.15ex);
        \fill<4->[shift={(-0.5ex,0.25ex)},rotate=100] (0,0) ellipse (0.15ex and
        0.15ex);
        %% mouth small
        \draw<1>[thick] (-0.5ex,-1ex)
        ..controls
        (-0.25ex,-1.25ex)and(0.25ex,-1.25ex)..(0.5ex,-1ex);
        %% mouth normal
        \draw<2>[thick] (-1.0ex,-1ex)..controls
        (-0.5ex,-1.5ex)and(0.5ex,-1.5ex)..(1ex,-1ex);
        %% mouth big
        \draw<3->[thick] (-1.5ex,-0.5ex)
        ..controls (-0.7ex,-1.7ex)and(0.7ex,-1.7ex)..(1.5ex,-0.5ex);
      \end{scope}
    \end{pgfonlayer}
    \uncover<1->{
      \node (availableat) at (4.5,3.1) {available at};
      \node (url) at (4.5,2.6)
           {\url{http://ultimate.informatik.uni-freiburg.de/smtinterpol}};
    }
    \begin{pgfonlayer}{back}
      \uncover<1>{
        \node[draw,shape=rectangle callout,fill=callOut,
          callout relative pointer={(-3.35,-3.05)},fit=(availableat)(url)]
             {};
      }
      \uncover<2->{
        \node[draw,shape=rectangle,fit=(availableat)(url),thick,fill=calledOut]{};
      }
    \end{pgfonlayer}
    \uncover<2->{
      \node (license) at (6,1.1) {licensed under LGPL version 3};
    }
    \begin{pgfonlayer}{back}
      \uncover<2>{
        \node[draw,shape=rectangle callout, callout relative
          pointer={(-4.15,-1.6)},fit=(license),fill=callOut]{};
      }
      \uncover<3->{
        \node[draw,shape=rectangle,thick,fit=(license),fill=calledOut]{};
      }
    \end{pgfonlayer}
    \uncover<3->{
      \node (java) at (6,-0.4) {platform independent (written in Java)};
    }
    \begin{pgfonlayer}{back}
      \uncover<3>{
        \node[draw,shape=rectangle callout, callout relative
          pointer={(-1.3,0)},fill=callOut,fit=(java)] {};
      }
      \uncover<4->{
        \node[draw,shape=rectangle,thick,fit=(java),fill=calledOut]{};
      }
    \end{pgfonlayer}
    \uncover<4>{
      \node[transform shape,scale=1.5,right] at (0,-3) {\Huge\leftthumbsup
        \large Thank you for
        your attention};
    }
  \end{tikzpicture}
\end{frame}

\end{document}
\begin{frame}
  \frametitle{Interpolation and Program Verification}

  \fbox{
%    \hbox to .05\textwidth{
  \begin{minipage}{.45\textwidth}
    \begin{alltt}
{\bf proc} f (x : int, y : int)\\
\hspace*{1em}{\bf returns} (r : int)\\
\hspace*{1em}{\bf requires} x $\geq$ 0\\
\hspace*{1em}{\bf ensures}  r $\geq$ y + 2 \{\\
\uncover<3->{\hspace*{2em}\{x $\geq$ 0\}}\\
\hspace*{2em}x := x + 2;\\
\uncover<3->{\hspace*{2em}\{x $\geq$ 2\}}\\
\hspace*{2em}x := x + y;\\
\uncover<2->{\hspace*{2em}\{x $\geq$ y + 2\}}\\
\hspace*{2em}r := x;\\
\uncover<3->{\hspace*{2em}\{r $\geq$ y + 2\}}\\
\}
    \end{alltt}
  \end{minipage}}~
  \begin{minipage}{.45\textwidth}
Single Static Assignment
\vspace{3.7ex}
\begin{align*}
   &x_0 \geq 0\\[-.6ex]
\uncover<3->{x_0 \geq 0}&\\[-.6ex]
   &x_1 = x_0 + 2\\[-.6ex]
\uncover<3->{x_1 \geq 2}&\\[-.6ex]
   &x_2 = x_1 + y_0\\[-.6ex]
\uncover<2->{x_2 \geq y_0+2}&\\[-.6ex]
   &r_3 = x_2\\[-1ex]
\uncover<3->{r_3 \geq y_0+2}&\\[-.6ex]
   &r_3 < y_0 + 2
\end{align*}
  \end{minipage}

  \bigskip
  \uncover<2->{Interpolants correspond to Hoare annotations}
\end{frame}

\begin{frame}[fragile]
  \frametitle{Inductive Sequence of Interpolants}
  \begin{minipage}{.4\textwidth}
    Inductive Interpolants:
    \[I_i \land \phi_i \Rightarrow I_{i+1}\]
    \[I_0 = \mathbf{true}, I_{n+1} = \mathbf{false}\]

    Symbol condition:
    \begin{align*}
    sym(I_i) \subseteq{} &
    sym(\bigwedge_{j<i} \phi_j)\\
    &{}\cap sym(\bigwedge_{j\geq i} \phi_j)
    \end{align*}

  \end{minipage}~
  \begin{minipage}{.55\textwidth}
\begin{align*}
\uncover<1->{I_0: \mathbf{true}}&\\[-.6ex]
   &\phi_0: x_0 \geq 0\\[-.6ex]
\uncover<1->{I_1: x_0 \geq 0}&\\[-.6ex]
   &\phi_1: x_1 = x_0 + 2\\[-.6ex]
\uncover<1->{I_2: x_1 \geq 2}&\\[-.6ex]
   &\phi_2: x_2 = x_1 + y_0\\[-.6ex]
\uncover<1->{I_3: x_2 \geq y_0+2}&\\[-.6ex]
   &\phi_3: r_3 = x_2\\[-1ex]
\uncover<1->{I_4: r_3 \geq y_0+2}&\\[-.6ex]
   &\phi_4: r_3 < y_0 + 2\\[-.6ex]
\uncover<1->{I_5: \mathbf{false}}&
\end{align*}
  \end{minipage}
\end{frame}

\begin{frame}
  \frametitle{Programs with Subprocedures}

  \fbox{
%    \hbox to .05\textwidth{
  \begin{minipage}{.45\textwidth}
    \begin{alltt}
$\vdots$\\
\hspace*{2em}{\color{ALUgreen}x} := {\color{ALUgreen}z};\\
\uncover<3->{\hspace*{2em}\{{\color{ALUgreen}x} $\geq$ {\color{ALUgreen}z}\}}\\
\hspace*{2em}{\color{ALUgreen}x} := add({\color{ALUgreen}x}, {\color{ALUgreen}y});\\
\uncover<3->{\hspace*{2em}\{{\color{ALUgreen}x} $\geq$ {\color{ALUgreen}y} + {\color{ALUgreen}z}\}}\\
\hspace*{2em}{\color{ALUgreen}r} := {\color{ALUgreen}x};\\
\uncover<3->{\hspace*{2em}\{{\color{ALUgreen}r} $\geq$ {\color{ALUgreen}y} + {\color{ALUgreen}z}\}}\\
\hspace*{2em}{\bf assert} {\color{ALUgreen}r} $\geq$ {\color{ALUgreen}y} + {\color{ALUgreen}z};\\
$\vdots$\\
{\bf proc} add ({\color{ALUblue}x} : int, {\color{ALUblue}y} : int)\\
\hspace*{1em}{\bf returns} ({\color{ALUblue}s} : int)\\
\uncover<3->{\hspace*{2em}\{{\color{ALUblue}x} $\geq$ {\color{ALUgreen}z} $\land$ {\color{ALUblue}y} $\geq$ {\color{ALUgreen}y}\}}\\
\hspace*{2em}{\color{ALUblue}s} := {\color{ALUblue}x} + {\color{ALUblue}y};\\
\uncover<3->{\hspace*{2em}\{{\color{ALUblue}s} $\geq$ {\color{ALUgreen}y} + {\color{ALUgreen}z}\}}\\
\}
    \end{alltt}
  \end{minipage}}~
  \begin{minipage}{.45\textwidth}
\begin{align*}
   &{\color{ALUgreen}x_1} = {\color{ALUgreen}z_0}\\[-.6ex]
\uncover<2->{{\color{ALUgreen}x_1} \geq {\color{ALUgreen}z_0}}&\\[-.6ex]
\text{\footnotesize(call)\quad}
&{\color{ALUblue}x_2} = {\color{ALUgreen}x_1} \land {\color{ALUblue}y_2} = {\color{ALUgreen}y_0}\\[-.6ex]
\uncover<2->{{\color{ALUblue}y_2}\geq {\color{ALUgreen}y_0} \land {\color{ALUblue}x_2} \geq {\color{ALUgreen}z_0}}&\\[-.6ex]
   &\quad {\color{ALUblue}s_3} = {\color{ALUblue}x_2} + {\color{ALUblue}y_2}
\\[-.6ex]
\uncover<2->{{\color{ALUblue}s_3} \geq {\color{ALUgreen}y_0}+{\color{ALUgreen}z_0}}&\\[-.6ex]
\text{\footnotesize(return)\quad}
   &{\color{ALUgreen}x_4} = {\color{ALUblue}s_3}\\[-1ex]
\uncover<2->{{\color{ALUgreen}x_4} \geq {\color{ALUgreen}y_0}+{\color{ALUgreen}z_0}}&\\[-.6ex]
   &{\color{ALUgreen}r_5} = {\color{ALUgreen}x_4}\\[-1ex]
\uncover<2->{{\color{ALUgreen}r_5} \geq {\color{ALUgreen}y_0}+{\color{ALUgreen}z_0}}&\\[-.6ex]
   &{\color{ALUgreen}r_5} < {\color{ALUgreen}y_0} + {\color{ALUgreen}z_0}
\end{align*}
  \end{minipage}
\end{frame}

\begin{frame}
  \frametitle{Nested Sequences of Interpolants}

  \begin{tikzpicture}
    \node (xz) at (0,0) {${\color{ALUgreen}x_1} = {\color{ALUgreen}z_0}$};
    \node (call) at (0,-1) {${\color{ALUblue}x_2} = {\color{ALUgreen}x_1}
    \land {\color{ALUblue}y_2} = {\color{ALUgreen}y_0}$};
    \node (s)  at(2,-2) {${\color{ALUblue}s_3} = {\color{ALUblue}x_2} + {\color{ALUblue}y_2}$};
    \node (ret)  at(0,-3) {${\color{ALUgreen}x_4} = {\color{ALUblue}s_3}$};
    \node (rx)  at(0,-4) {${\color{ALUgreen}r_5} = {\color{ALUgreen}x_4}$};
    \node (assert)  at(0,-5) {${\color{ALUgreen}r_5} < {\color{ALUgreen}y_0} + {\color{ALUgreen}z_0}$};
    \draw[thick, ->] (xz) -- (call);
    \draw[thick, ->] (call) -- node (calledge) {} (s);
    \draw[thick, ->] (s) -- (ret);
    \draw[thick, ->] (ret) -- (rx);
    \draw[thick, ->] (rx) -- (assert);
    \uncover<2->{\draw[line width=3pt, color=black!50,->] (call) -- (ret);}
    \uncover<2->{
      \draw[shift=(calledge),color=ALUred, scale=3,very thick] plot[mark=x] coordinates {(0,0)};
      \uncover<2>{
        \node[align=center,text width=2.5cm] (expl) at (6,-1) {Ensure locality of interpolants of subprocedure};
        \draw[->] (expl) to (calledge);
      }
    }
    \uncover<3->{
      \node[right, outer sep=3mm] (ixz) at (xz.east) {${\color{ALUgreen}x_1} \geq {\color{ALUgreen}z_0}$};
      \node[right, outer sep=3mm] (icall) at (call.east) {${\color{ALUblue}x_2} \geq {\color{ALUgreen}z_0} \land {\color{ALUblue}y_2} > {\color{ALUgreen}y_0}$};
      \node[right, outer sep=3mm] (is) at (s.east) {${\color{ALUblue}s_3} \geq {\color{ALUblue}x_2} + {\color{ALUblue}y_2}$};
      \node[right, outer sep=3mm] (iret) at (ret.east) {${\color{ALUgreen}x_4} \geq {\color{ALUgreen}y_0} + {\color{ALUgreen}z_0}$};
      \node[right, outer sep=3mm] (irx) at (rx.east) {${\color{ALUgreen}r_5} \geq {\color{ALUgreen}y_0} + {\color{ALUgreen}z_0}$};
      \node[right, outer sep=3mm] (iassert) at (assert.east) {$\textbf{false}$};
    }
  \end{tikzpicture}
\end{frame}

\begin{frame}
  \frametitle{Corresponding Hoare Proof}

  \fbox{
%    \hbox to .05\textwidth{
  \begin{minipage}[t]{.45\textwidth} 
   \begin{alltt}
$\vdots$\\
\hspace*{2em}{\color{ALUgreen}x} :=
        {\color{ALUgreen}z};\\ \uncover<2->{\hspace*{2em}\{{\color{ALUgreen}x}
          $\geq$ {\color{ALUgreen}z}\}}\\
\hspace*{2em}{\color{ALUgreen}x} := add({\color{ALUgreen}x},
        {\color{ALUgreen}y});\\ \uncover<2->{\hspace*{2em}\{{\color{ALUgreen}x}
          $\geq$ {\color{ALUgreen}y} + {\color{ALUgreen}z}\}}\\
\hspace*{2em}{\color{ALUgreen}r} :=
        {\color{ALUgreen}x};\\ \uncover<2->{\hspace*{2em}\{{\color{ALUgreen}r}
          $\geq$ {\color{ALUgreen}y} + {\color{ALUgreen}z}\}}\\
\hspace*{2em}{\bf assert} {\color{ALUgreen}r} $\geq$
        {\color{ALUgreen}y} + {\color{ALUgreen}z};\\ $\vdots$\\ {\bf
          proc} add ({\color{ALUblue}x} : int, {\color{ALUblue}y} :
        int)\\
\hspace*{1em}{\bf returns} ({\color{ALUblue}s} : int)\\ \\
\hspace*{2em}{\color{ALUblue}s} := {\color{ALUblue}x} +
        {\color{ALUblue}y};\\ \uncover<2->{\hspace*{2em}\{{\color{ALUblue}s}
          $\geq$ {\color{ALUblue}x} + {\color{ALUblue}y}\}}\\ \}
    \end{alltt}
  \end{minipage}}
  \begin{minipage}[t]{.45\textwidth}
    \vbox{}
  \begin{tikzpicture}[xscale=.8,yscale=1.2]\footnotesize
    \node (xz) at (0,0) {${\color{ALUgreen}x_1} =
      {\color{ALUgreen}z_0}$}; \node (call) at (0,-1)
          {${\color{ALUblue}x_2} = {\color{ALUgreen}x_1} \land
            {\color{ALUblue}y_2} = {\color{ALUgreen}y_0}$}; \node (s)
          at(2,-2) {${\color{ALUblue}s_3} = {\color{ALUblue}x_2} +
            {\color{ALUblue}y_2}$}; \node (ret) at(0,-3)
          {${\color{ALUgreen}x_4} = {\color{ALUblue}s_3}$}; \node (rx)
          at(0,-4) {${\color{ALUgreen}r_5} = {\color{ALUgreen}x_4}$};
          \node (assert) at(0,-5) {${\color{ALUgreen}r_5} <
            {\color{ALUgreen}y_0} + {\color{ALUgreen}z_0}$};
          \draw[thick, ->] (xz) -- (call); \draw[thick, ->] (call) --
          node (calledge) {} (s); \draw[thick, ->] (s) -- (ret);
          \draw[thick, ->] (ret) -- (rx); \draw[thick, ->] (rx) --
          (assert); \draw[line width=3pt, color=black!50,->] (call) --
          (ret); \draw[shift=(calledge),color=ALUred, scale=3,very
            thick] plot[mark=x] coordinates {(0,0)}; \node[right,
            outer sep=3mm] (ixz) at (xz.east) {${\color{ALUgreen}x_1}
            \geq {\color{ALUgreen}z_0}$}; \node[right, outer sep=3mm]
          (icall) at (call.east) {${\color{ALUblue}x_2} \geq
            {\color{ALUgreen}z_0} \land {\color{ALUblue}y_2} >
            {\color{ALUgreen}y_0}$}; \node[right, outer sep=3mm] (is)
          at (s.east) {${\color{ALUblue}s_3} \geq {\color{ALUblue}x_2}
            + {\color{ALUblue}y_2}$}; \node[right, outer sep=3mm]
          (iret) at (ret.east) {${\color{ALUgreen}x_4} \geq
            {\color{ALUgreen}y_0} + {\color{ALUgreen}z_0}$};
          \node[right, outer sep=3mm] (irx) at (rx.east)
               {${\color{ALUgreen}r_5} \geq {\color{ALUgreen}y_0} +
                 {\color{ALUgreen}z_0}$}; \node[right, outer sep=3mm]
               (iassert) at (assert.east) {$\textbf{false}$};
  \end{tikzpicture}
  \end{minipage}
\end{frame}


\begin{frame}[fragile]
  \frametitle{Nested Interpolants (a.k.a. Tree Interpolants)}

  \begin{tikzpicture}
    \node (xz) at (0,0) {$\phi_1$};
    \node (call) at (0,-1) {$\phi_2$};
    \node (s)  at(2,-2) {$\phi_3$};
    \node (ret)  at(0,-3) {$\phi_4$};
    \node (rx)  at(0,-4) {$\phi_5$};
    \node (assert)  at(0,-5) {$\phi_6$};
    \draw[thick, ->] (xz) -- (call);
%    \draw[thick, ->] (call) -- node (calledge) {} (s);
    \draw[thick, ->] (s) -- (ret);
    \draw[thick, ->] (ret) -- (rx);
    \draw[thick, ->] (rx) -- (assert);
    \draw[line width=3pt, color=black!50,->] (call) -- (ret);

    \node[right, outer sep=3mm] (ixz) at (xz.east) {$\{I_1\}$};
    \node[right, outer sep=3mm] (icall) at (call.east) {$\{I_2\}$};
    \node[right, outer sep=3mm] (is) at (s.east) {$\{I_3\}$};
    \node[right, outer sep=3mm] (iret) at (ret.east) {$\{I_4\}$};
    \node[right, outer sep=3mm] (irx) at (rx.east) {$\{I_5\}$};
    \node[right, outer sep=3mm] (iassert) at (assert.east) {$\{I_6\}$};
    
    \uncover<2->{
    \node[right] at(5,-1.5) {
      $\begin{array}{rcl}
        \phi_1 &\Rightarrow& I_1\\
        I_1\land \phi_2 &\Rightarrow& I_2\\
        \phi_3 &\Rightarrow& I_3\\
        I_2 \land I_3 \land \phi_4 &\Rightarrow& I_4\\
        I_4 \land \phi_5 &\Rightarrow& I_5\\
        I_5 \land \phi_6 &\Rightarrow& I_6\\
        I_6 &=& \textbf{false}
      \end{array}$
    };
    }
    
    \node[right,align=left] at(3,-5) {
      \noindent Inductivity:\\
      $\quad\bigwedge \{I_j \mid j \text{ child of } i\} \land \phi_i \Rightarrow I_i$\\[2ex]
      Symbol condition:\\
      $\ \begin{array}{rl}
        sym(I_i) \subseteq &
        sym\{\phi_j \mid \text{$j$ in subtree with root $i$}\} \\
        \cap &
        sym\{\phi_j \mid \text{$j$ not in subtree with root $i$}\}
      \end{array}$};
  \end{tikzpicture}
\end{frame}

\begin{frame}[fragile]
  \frametitle{How to compute Nested Interpolants}
  \begin{itemize}
  \item Na{\"\i}ve Method: \\
    $I_i = \textit{interpolate\/}(
    \bigwedge \{\phi \mid \phi\text{ in subtree }i\}, 
    \bigwedge \{\phi \mid \phi\text{ not in subtree }i\});$ 
    \pause
    \begin{itemize}
    \item does not guarantee inductivity (unsound).
    \end{itemize}
    \pause
  \item Method presented in [Heizmann et al. 2010]\\
    $\begin{array}{r@{}l}I_i = \textit{interpolate\/}(&
    \phi_i \land \bigwedge \{I_j \mid j\text{ child of }i\}, \\
    &\bigwedge \{I_j \mid j\text{ call parent of }i\} \land
    \bigwedge \{\phi_j \mid j > i\})\end{array}$
    \pause
    \begin{itemize}
    \item needs a new proof for every interpolant
    \end{itemize}
    \pause
  \item Method we propose for SMTInterpol\\
    Compute all nested interpolants directly from a single proof tree.
  \end{itemize}
\end{frame}


\begin{frame}[fragile]
  \frametitle{Proof trees in SMTInterpol}
  \begin{tikzpicture}[level distance=10mm,
      level/.style={sibling distance=60mm/#1}]

    \node (bottom) at (0,0) {\textbf{false}} [grow'=up]
       child { node (c1) {$C^{(7)}$}
         child { node (c11) {$C^{(3)}$}
           child { node (c111) {${\color{ALUblue}C^{(1)}}$} }
           child { node (c112) {${\color{ALUgreen}C^{(2)}}$} }
         }
         child { node (c12) {${\color{ALUgreen}C^{(4)}}$} }
       }
       child { node (c2) {$C^{(8)}$} 
         child { node (c21) {${\color{ALUblue}C^{(5)}}$} }
         child { node (c22) {${\color{ALUblue}C^{(6)}}$} }
       };
    \node at (0,0) {\textbf{false}};

    \uncover<3>{
      \node (tl) at (1,4) {\color{ALUblue}theory lemmas};
      \draw[color=black!50,line width=1pt] (tl) -- (c111); 
      \draw[color=black!50,line width=1pt] (tl) -- (c21);
      \draw[color=black!50,line width=1pt] (tl) -- (c22);
    }

    \uncover<2>{
      \node (if) at (-3,4) {\color{ALUgreen}input formulas};
      
      \draw[color=black!50,line width=1pt] (if) -- (c112);
      \draw[color=black!50,line width=1pt] (if) -- (c12);
    }
    
    \uncover<4>{
      \node[align=center] (rr) at (3,3.5) {resolution rule\\
        $\frac{\ell, C \quad \bar\ell, C'}{C,C'}$};
      
      \draw[color=black!50,line width=1pt] (bottom)+(0,.25) -- (rr);
      \draw[color=black!50,line width=1pt] (c1)+(0,.25) -- (rr);
      \draw[color=black!50,line width=1pt] (c2)+(0,.25) -- (rr);
      \draw[color=black!50,line width=1pt] (c11)+(0,.25) -- (rr);
    }

    \begin{scope}[outer sep=0mm,inner sep=0mm]
    \uncover<5-> {
      \node[right] at (c111.east) {\tiny$\{I_1,\dots,I_n\}$};
      \node[right] at (c112.east) {\tiny$\{I_1,\dots,I_n\}$};
      \node[right] at (c12.east) {\footnotesize$\{I_1,\dots,I_n\}$};
      \node[right] at (c21.east) {\footnotesize$\{I_1,\dots,I_n\}$};
      \node[right] at (c22.east) {\footnotesize$\{I_1,\dots,I_n\}$};
    }
    \uncover<6-> {
      \node[right] at (c1.east) {$\{I_1,\dots,I_n\}$};
      \node[right] at (c11.east) {\footnotesize$\{I_1,\dots,I_n\}$};
      \node[right] at (c2.east) {$\{I_1,\dots,I_n\}$};
      \node[right] at (bottom.east) {$\{I_1,\dots,I_n\}$};
    }
    \end{scope}
  \end{tikzpicture}
  
  \begin{itemize}
  \item<5-> Compute \emph{Partial} Nested Interpolants for every leaf.\\
  \item<6-> Propagate Partial Interpolants to the bottom clause.
  \end{itemize}
\end{frame}

\begin{frame}
  \frametitle{Definition: Partial Nested Interpolants}
  
  Given Input Formulas partitioned into $\phi_1,\dots, \phi_n$, proof tree.

  \begin{itemize}
  \item Every Clause $C$ in the proof tree follows from the input:
    \[\phi_1 \land \dots \land \phi_n \land \lnot C \text{ is unsat}\]
  \item $\lnot C$ is a conjunction of literals.
  \item For simplicity, assume every literal belongs to one part.\\
    Split $\lnot C$ into the parts $\lnot C_1\land \dots \land \lnot C_n$.\\
  \end{itemize}
  \pause
  \begin{definition}[Partial Nested Interpolant]
    The \emph{Partial Nested Interpolant} for $C$ and the input
    $\phi_1,\dots,\phi_n$ is the Nested Interpolant for
    \[ \phi_1\land \lnot C_1, \dots, \phi_n \land \lnot C_n \]
  \end{definition}
\end{frame}

\begin{frame}
  \frametitle{Computing Partial Nested Interpolants}
  
  \begin{itemize}
  \item {\color{ALUblue} Theory lemmas}:\\
    Theory-specific interpolation procedure.\\
    Basically, $I_i$ summarises the (in)equalities of subtree $i$.
    \pause
  \item {\color{ALUgreen} Input formula} $C$:
    \[I_i = \begin{cases}
      \bigvee\{C_j \mid j\text{ not in subtree }i\} & \text{if $C$ in subtree $i$}\\
      \bigwedge\{\lnot C_j \mid j\text{ in subtree }i\} & \text{if $C$ not in subtree $i$}
    \end{cases}\]
    \pause
  \item Resolution on $\ell$ of $C^{(1)}~\{I^{(1)}_1,\dots,I^{(1)}_n\}$ and 
    $C^{(2)}~\{I^{(2)}_1,\dots,I^{(2)}_n\}$:
    \[I_i = \begin{cases}
      I^{(1)}_i \lor I^{(2)}_i & \text{if $\ell$ in subtree $i$}\\
      I^{(1)}_i \land I^{(2)}_i & \text{if $\ell$ not in subtree $i$}
    \end{cases}\] 
    \pause
  \item Correctness can be proved by induction over the proof tree.
  \end{itemize}
\end{frame}

\begin{frame}
  \frametitle{Proposed Syntax for SMTLIB 2.0}
  Goal: Nice encoding for sequences and nested sequences.
  \bigskip

  \pause
  Example:
  \begin{tikzpicture}[rotate=90]
    \node (xz) at (0,0) {$\phi_1$};
    \node (call) at (0,-1) {$\phi_2$};
    \node (s)  at(1,-2) {$\phi_3$};
    \node (rx)  at(1,-3) {$\phi_4$};
    \node (ret)  at(0,-4) {$\phi_5$};
    \node (assert)  at(0,-5) {$\phi_6$};
    \draw[thick, ->] (xz) -- (call);
%    \draw[thick, ->] (call) -- node (calledge) {} (s);
    \draw[thick, ->] (s) -- (rx);
    \draw[thick, ->] (rx) -- (ret);
    \draw[thick, ->] (ret) -- (assert);
    \draw[line width=1.5pt, color=black!50,->] (call) -- (ret);
  \end{tikzpicture}
  \begin{alltt}
  (get-interpolants phi1 phi2 (phi3 phi4) phi5 phi6)
  \end{alltt}

  \pause
  This can be parsed by this simple grammar:
  \begin{alltt}
  tree~~~~~::= named-formula | children named-formula \\
  children ::= tree ( children ) | tree \\
  \end{alltt}

  \pause
  We return the interpolants as a flat sequence.\\
  The last  interpolant, which is always $\textbf{false}$, is omitted.
\end{frame}

\begin{frame}
  \frametitle{Conclusion}

  \begin{itemize}
  \item Inductive Sequences of Interpolants are not suited for programs with procedures.
  \item Nested Sequences of Interpolants are the solution.
  \item Nested Interpolants are Tree Interpolants.
  \item We will add support for Nested Interpolants in SMTInterpol.
  \end{itemize}
\end{frame}

\end{document}
